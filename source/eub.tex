\section{When $\bar{f}^+$ is not eventually constant}


In this section we will take a look at the situation where the upward projection $\bar{f}^+$ of $\bar{f}$ into some product
$\prod S(a)$ fails to be eventually constant.

This looks to divide into two cases:

\Circled{1}  There is a club $E\subseteq\lambda$ such that $\alpha<\beta$ in $C$ implies $f^+_\alpha<_I f^+_\beta$.

\Circled{2}  There is an $\alpha<\lambda$ such that  for all $\beta<\lambda$,
\begin{equation}
\alpha<\beta\Longrightarrow \{a\in A: f^+_\alpha(a) = f^+_\beta(a)\}\in I^+,
\end{equation}
and
\begin{equation}
(\forall \beta<\lambda)(\exists \gamma<\lambda)\left[\{a\in A: f^+_\beta(a)<f^+_\gamma(a)\}\in I^+\right]
\end{equation}

\begin{proposition}
Suppose $X\subseteq\lambda$ is of order-type $\tau$ for some regular $\tau$ greater than $|A|$ and each $|S(a)|$. Then there is a
$\beta\in\nacc(X)$ such that $\bar{f}^+$ is bounded mod $I$ by the function
\begin{equation}
g = \sup\{f^+_\alpha:\alpha\in X\cap\beta\}.
\end{equation}
\end{proposition}
\begin{proof}
Suppose $X\subseteq \lambda$ is of order-type $\tau$, but for every $\beta\in\nacc(X)$, there is a $\gamma<\lambda$ such that
\begin{equation}
\{a\in A: \sup\{f^+_\alpha:\alpha\in X\cap\beta\} < f^+_\gamma(a)\}\notin I.
\end{equation}
By passing to a thinner set, we can arrange that $\gamma = \beta$, that is,
\begin{equation}
(\forall \beta\in\nacc(X))\left[\{a\in A: \sup\{f_\gamma(a):\gamma\in X\cap\beta\}<f^+_\beta(a)\}\notin I\right]
\end{equation}
In particular, for each $\beta\in\nacc(X)$ we can choose $a_\beta\in A$ least such that
\begin{equation}
\sup\{f_\gamma(a_\beta):\gamma\in X\cap\beta\}<f_\beta(a_\beta).
\end{equation}
Since $|A|<\tau$, it follows that there is a single $a^*\in A$ and unbounded $Y\subseteq \nacc(X)$ such that $a_\beta$ is $a^*$
for all $\beta\in X$ and therefore $\langle f_\beta(a)

\end{proof}

\begin{claim}
There is no set $X\subseteq\lambda$ such that $\otp(X)=\max\{|A|^+,\tau^+\}$ and
\begin{equation}
(\forall\alpha\in X)\left[ f^+_{\min(X\setminus \alpha+1)}\nleq_I \sup\{f^+_\gamma:\gamma\in X\cap(\alpha+1)\}\right]
\end{equation}
\end{claim}
\begin{proof}
By way of contradiction, assume that such an $X$ exists.  Our assumption says that for $\alpha\in X$, if $\beta$ is the next
element beyond $\alpha$ in $X$, then
\begin{equation}
\left\{a\in A: \sup\{f^+_\gamma(a):\gamma\in X\cap\beta\}<f^+_\beta(a)\right\}\in I^+.
\end{equation}
So this means for each $\beta\in \nacc(X)$, we can choose $a_\beta\in A$ such that
\begin{equation}
(\forall \gamma\in X\cap\beta)\left[ f^+_\gamma(a_\beta)<f^+_\beta(a_\beta)\right]
\end{equation}
But $|X|>|A|$, so there is a cofinal $Y\subseteq X$ and $a^*\in A$ such that
\begin{equation}
\beta\in Y\Longrightarrow a_\beta = a^*.
\end{equation}
Now the sequence $\langle f^+_\beta(a^*):\beta\in Y\rangle$ is strictly increasing in $S(a^*)$, and we have a contradiction as
\begin{equation}
|S(a^*)|\leq \tau < |Y|.
\end{equation}
\end{proof}

So this leads us to the following question:  Given our sequence $\bar{f}$, is there some way to guarantee that whenever we build
a sequence of upper bounds for $\bar{f}$, the resulting sequence $\langle f^+_\alpha:\alpha<\lambda\rangle$ contains a set $X$ as
in the claim?  This would ensure that the sequence $\langle f^+_\alpha:\alpha<\lambda\rangle$ must stabilize.


\section{Step 2}

